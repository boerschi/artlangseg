% File acl2014.tex

\documentclass[11pt]{article}
\usepackage{acl2014}
\usepackage{times}
\usepackage{url}
\usepackage{latexsym}

%\setlength\titlebox{5cm}

% You can expand the titlebox if you need extra space
% to show all the authors. Please do not make the titlebox
% smaller than 5cm (the original size); we will check this
% in the camera-ready version and ask you to change it back.

\title{The effects of lexical phonotactics, saturation,
  and frequency skew on segmentability}

\author{Benjamin B{\"o}rschinger \\
    Department of Computing \\
    Macquarie University \\\And
  Robert Daland \\
    Linguistics \\
    UCLA \\
    {\tt benjamin.borschinger@mq.edu.au} \\\And
  Abdellah Fourtassi \\
    LCSP \\
    ENS/EHESS/CNRS \\
    \\
    {\tt \{r.daland\,abdellah.fourtassi\,emmanuel.dupoux\}@gmail.com} \\\And
  Emmanuel Dupoux \\
    LCSP \\
    ENS/EHESS/CNRS }

\date{}

\begin{document}
\maketitle
\begin{abstract}
  Previous works have proposed that the `segmentability' of a language
  depends on its phonotactic structure and can be measured as the
  entropy of licit segmentations of a corpus sample. These proposals
  are tested here by generating artificial languages and measuring
  their segmentability. Maximally permissive and restrictive grammars
  (Pseudo-Berber, Pseudo-Senufo) were used to generate corpus samples 
  in which the lexical saturation and frequency distributions were
  varied parametrically. Lexical saturation tends to cause oversegmentation.
  Interestingly, better segmentation was found for flat frequency
  distributions than for the Zipfian distributions that match the
  prior assumptions of the segmenter. The results show heretofore
  unsuspected nuances of the relationship between phonotactic complexity,
  word length, and word segmentation.
\end{abstract}


\section{Introduction}

Word segmentation is the perceptual process by which infant and adult listeners parse the continuous speech signal into a sequence of discrete, word-like units. The acquisition of word segmentation has been the subject of intense computational scrutiny, where it is typically operationalized as the unsupervised partitioning of a phonemically transcribed, child-directed corpus, and assessed against an orthographically derived gold standard~\cite{GGJ09,DP11,PearlEtAl11}.

Recent years have seen a number of crosslinguistic modeling studies converging on the same conclusion. Across a typologically diverse range of languages (e.g. Arabic, Japanese, Korean, Russian, and Spanish), and across a range of models (lexical, phonotactic, and hybrid), better segmentation is unfailingly predicted for English than for other languages \cite{Fleck08,Daland09,DP11,FourtassiEtAl13,DalandZuraw13}. One possible way to interpret this fact is as evidence of the insufficiency of the models. After all, the developmental trajectory of language acquisition is empirically known to be remarkably robust to cross-linguistic variation in related domains like phonetic categorization~\cite{ref} and meter/prosody/accentuation~\cite{ref}. On the other hand, there is considerably less cross-linguistic work on word segmentation in infants, and the available work is consistent with the alternative hypothesis, e.g. later segmentation of function words by French-acquiring infants than English-acquiring ones.~\cite{ref}. On balance, then, existing work suggests that languages genuinely differ in segmentability (i.e. the predicted ease/accuracy with which infants might learn to segment speech in that language).

Recently, it was proposed in two independent studies that cross-linguistic differences in segmentability may arise in part or in whole from language-specific phonotactics. Daland and Zuraw~\shortcite{DalandZuraw13} investigated the segmentability of Korean using a phonotactic segmentation model~\cite{DP11}. Korean possesses many edge-sensitive phonological processes, so one might naturally think that segment sequences would be distinct across word boundaries versus within words; the authors found that overall, this didn't hold. For example, vowel-vowel (hiatus) sequences are tolerated within words in Korean, while vowels are also common both word-initially and in word-final inflections; thus hiatus is not a good boundary cue in Korean (though it is in English). Fourtassi et al.~\shortcite{FourtassiEtAl13} investigated the segmentability of Japanese using a lexical segmentation model~\cite{GGJ09}. They found ...


\section{Methods}

this section describes what we actually did

\section{Results}

this section describes what we found

\begin{table}[h]
\begin{center}
\begin{tabular}{|c||c|cccccc|}
  \hline
  Language & *Struct & brent & zipf16 & zipf12 & linear & flat & point \\
  \hline
  Senufo & 1 & .85 & .65 & .83 & .66 & .58 & .13 \\
  Senufo & 2 & .87 & .65 & .83 & .78 & .76 & .13 \\
  Senufo & 3 & .83 & .50 & .71 & .91 & .91 & .13 \\
  \hline
  Berber & 1 & .93 & .85 & .91 & .70 & .62 & .20 \\
  Berber & 2 & .95 & .75 & .92 & .78 & .73 & .21 \\
  Berber & 3 & .95 & .62 & .87 & .93 & .91 & .20 \\
  \hline
\end{tabular}
\end{center}
\caption{\label{Results.}}
\end{table}

\section{Discussion}

this section says what it all means

{\bf Citations}: Citations within the text appear in parentheses
as~\cite{Gusfield:97} or, if the author's name appears in the text
itself, as Gusfield~\shortcite{Gusfield:97}.  Append lowercase letters
to the year in cases of ambiguity.  Treat double authors as
in~\cite{Aho:72}, but write as in~\cite{Chandra:81} when more than two
authors are involved. Collapse multiple citations as
in~\cite{Gusfield:97,Aho:72}.

If you are using the provided \LaTeX{} and Bib\TeX{} style files, you
can use the command \verb|\newcite| to get ``author (year)'' citations.

\textbf{Please do not use anonymous citations} and do not include
acknowledgements when submitting your papers. Papers that do not
conform to these requirements may be rejected without review.

% include your own bib file like this:
%\bibliographystyle{acl}
%\bibliography{acl2014}

\begin{thebibliography}{}

\bibitem[\protect\citename{Pearl, Goldwater, and Steyvers, 2011}]{PearlEtAl11}
Lisa Pearl, Sharon Goldwater, and Mark Steyvers.
\newblock 2011.
\newblock {\em Research on Language and Computation}, 8(2):107--132.
\newblock DOI 10.1007/s11168-011-9074-5.

\bibitem[\protect\citename{Goldwater, Griffiths, and Johnson, 2009}]{GGJ09}
Sharon Goldwater, Thomas L. Griffiths, and Mark Johnson.
\newblock 2009.
\newblock {\em Cognition}, 112:21--54.

\bibitem[\protect\citename{Daland and Pierrehumbert, 2011}]{DP11}
Robert Daland and Janet~B. Pierrehumbert.
\newblock 2011.
\newblock {\em Cognitive Science}, 35(1):119--155.

\bibitem[\protect\citename{what appears in the text}]{how you label it}
Authors.
\newblock date.
\newblock {\em to italicize}, deets.

\bibitem[\protect\citename{what appears in the text}]{how you label it}
Authors.
\newblock date.
\newblock {\em to italicize}, deets.

\bibitem[\protect\citename{what appears in the text}]{how you label it}
Authors.
\newblock date.
\newblock {\em to italicize}, deets.


\bibitem[\protect\citename{Aho and Ullman}1972]{Aho:72}
Alfred~V. Aho and Jeffrey~D. Ullman.
\newblock 1972.
\newblock {\em The Theory of Parsing, Translation and Compiling}, volume~1.
\newblock Prentice-{Hall}, Englewood Cliffs, NJ.

\bibitem[\protect\citename{{American Psychological Association}}1983]{APA:83}
{American Psychological Association}.
\newblock 1983.
\newblock {\em Publications Manual}.
\newblock American Psychological Association, Washington, DC.

\bibitem[\protect\citename{{Association for Computing Machinery}}1983]{ACM:83}
{Association for Computing Machinery}.
\newblock 1983.
\newblock {\em Computing Reviews}, 24(11):503--512.

\bibitem[\protect\citename{Chandra \bgroup et al.\egroup }1981]{Chandra:81}
Ashok~K. Chandra, Dexter~C. Kozen, and Larry~J. Stockmeyer.
\newblock 1981.
\newblock Alternation.
\newblock {\em Journal of the Association for Computing Machinery},
  28(1):114--133.

\bibitem[\protect\citename{Gusfield}1997]{Gusfield:97}
Dan Gusfield.
\newblock 1997.
\newblock {\em Algorithms on Strings, Trees and Sequences}.
\newblock Cambridge University Press, Cambridge, UK.

\end{thebibliography}

\end{document}
